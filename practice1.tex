\documentclass[11pt]{amsart}

\usepackage{geometry}                
\geometry{letterpaper}                   

\usepackage{amsfonts}
\usepackage{amsmath}
\usepackage{amssymb}

\title{LaTeX Nw}
\author{Raul Villalobos}
\date{July 5, 2025}

\begin{document}
\maketitle
Let $a$,~$b \in \mathbb{Z}$ \\
Let $\pi = a/b$ \\
Define polynomials: \\
\begin{center}
$f(x) = \frac{x^{n}(a-bx)^n}{n!}$, \\
\vspace{4pt}
$F(x) = f(x) - f^{(2)}(x)+f^{(4)}(x) - \cdots +(-1)^{n}f^{(2n)}(x)$
\vspace{4pt}
\end{center}

$n$ is some postitive integer. Since $n!f(x)$ has integral coefficients and terms in $x$ of degree not less than $n$,~$f(x)$ and its derivatives $f^{(i)}(x)$ have integral values for $x=0;$ also for $x=\pi=a/b$,~since$f(x)=f(a/b-x)$. We have \\
\begin{center}
$\frac{d}{dx} \{F'(x)\sin x -F(x)\cos x\} = F''(x)\sin x + F(x) \sin x = f(x) \sin x$ \\
\vspace{4pt}
\end{center}
and
\begin{equation}
  \int_{0}^{\pi} f(x) \sin x dx = [F'(x) \sin x - F(x) \cos x]_{0}^{\pi} = F(\pi) + F(0).
\end{equation}
$F(\pi)+F(0)$ is now an integer, since $f^{(i)}(\pi)$and$f^{(i)}(0)$ are integers. For $0<x<\pi$ though,
\begin{center}
$0<f(x)\sin x <\frac{\pi^{n}a^{n}}{n!}$,
\end{center}
so the integral from $(1)$ is $positive, but~arbitrarily~small$ for $n$ which is large enough. So $(1)$ is false, and our assumption that $\pi$ is rational is aswel.
\vspace{4pt}
Full proof by Ivan Niven\cite{Niven}.

\begin{thebibliography}{1}
\bibitem{Niven} I.~Niven, ``A simple proof that $\pi$ is irrational,''
\emph{Bull.\ Amer.\ Math.\ Soc.\ (New Series)}, vol.~53, no.~6, p.~509, June~1947.
\end{thebibliography}

\end{document}  